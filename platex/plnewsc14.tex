%% <2020-02-02> and <2020-04-12>
\documentclass{plnews}

\publicationyear{2020}% 発行年
\publicationmonth{04}% 発行月
\publicationissue{c14}% 番号
\author{日本語\TeX{}開発コミュニティ(\texttt{https://texjp.org/})}

\def\cs#1{\texttt{\char92\nobreak #1}}
\def\pTeX{p\kern-.15em\TeX}
\def\eTeX{$\varepsilon$-\TeX}
\def\epTeX{$\varepsilon$-\pTeX}
\def\pLaTeX{p\kern-.05em\LaTeX}
\def\pLaTeXe{p\kern-.05em\LaTeXe}
\xspcode`\\=1

\begin{document}

\maketitle

コミュニティ版\pLaTeXe\ \texttt{<2020-02-02>}および
\texttt{<2020-04-12>}について、
\pLaTeXe\ \texttt{<2019-10-01>}からの更新箇所をまとめます。


\section{\LaTeXe\ \texttt{<2020-02-02>}対応}
\LaTeXe\ \texttt{<2020-02-02>}で
NFSS(フォント選択の仕組み)が大幅に拡張されたことを受け、
\pLaTeXe\ \texttt{<2020-02-02>}で対応を施しました。
シリーズ・シェイプの多軸化は\file{fontaxes}から、
ファミリごとの実シリーズ値は\file{mweights}から、
それぞれアイデアを取り入れたようです。
\file{ltnews31}も参照してください。

\pLaTeXe{}での拡張部分は以下のとおりです。

\paragraph{シリーズの多軸化}
\begin{itemize}
 \item |\kanjiseries|, |\romanseries| (, |\fontseries|)\\
   |\DeclareFontSeriesChangeRule|による
   シリーズ更新規則に従うようにした。
 \item |\kanjiseriesforce|, |\romanseriesforce|\\
   無条件にシリーズを更新する(新規)。\\
   |\fontseriesforce|は和欧文両方に影響。
\end{itemize}

\paragraph{シェイプの多軸化}
\begin{itemize}
 \item |\kanjishape|, |\romanshape| (, |\fontshape|)\\
   |\DeclareFontShapeChangeRule|による
   シェイプ更新規則に従うようにした。
 \item |\kanjishapeforce|, |\romanshapeforce|\\
   無条件にシェイプを更新する(新規)。\\
   |\fontshapeforce|は和欧文両方に影響。
\end{itemize}

\paragraph{総称ファミリごとの実シリーズ値の設定}
\begin{itemize}
 \item |\DeclareFontSeriesDefault|\\
   第一引数(オプション)の総称ファミリ名に
   |rm|, |sf|, |tt|に加え|mc|, |gt|も指定できるようにした。
\end{itemize}

\paragraph{強調書体指定の一般化}
\begin{itemize}
 \item \pLaTeX{}で再定義している|\em| (, |\emph|)でも
   |\DeclareEmphSequence|による
   入れ子の強調書体のカスタマイズ機能をサポートした。
\end{itemize}

\paragraph{既定値}
\LaTeXe{}に追随して\pLaTeXe{}でも調整。

まず、\LaTeXe{}での既定値の変更を見てみます。
\begin{itemize}
 \item |\bfdefault|: |bx| → |b| に変更
 \item |\updefault|: |n| → |up| に変更
 \item |\shapedefault|: |\updefault| → |n| に変更
\end{itemize}
このうち「|bx| → |b|」は
太字を原則|b| (bold)に変更することを意味します。
これだけでは互換性に関して懸念がありますが、
\LaTeXe{}では同時に
「総称ファミリごとの実シリーズ値の設定」の新機能を活用して
\begin{itemize}
 \item Computer ModernとLatin Modernの場合だけは
   従来どおり太字を|bx| (bold-extended)に維持
\end{itemize}
という挙動を実現しています。

\pLaTeXe{}では、|\bfdefault|と|\updefault|について
\LaTeXe{}の設定がそのまま和文にも引き継がれます。
残る一つは和文用の既定値が別に存在します。
\begin{itemize}
 \item |\kanjishapedefault|: |\updefault| → |n| に変更
\end{itemize}

さて、和文の太字は従来|\bfdefault|一択(結果的に|bx|)
でしたが、\pLaTeXe\ \texttt{<2020-02-02>}では
「総称ファミリごとの実シリーズ値の設定」の新機能が
和文ファミリ|mc|, |gt|にも使えます。
ここで、|mc|, |gt|の太字をどう設定するかは議論の余地があります:
\begin{enumerate}
 \item 互換性のため、従来どおり|bx|を維持する。
 \item Computer ModernでもLatin Modernでもないので、
   |b|に変更する。
\end{enumerate}
この2案のうち、現在の版では2.すなわち
\begin{itemize}
 \item 和文ファミリ|mc|と|gt|の太字も|b| (bold)に変更
\end{itemize}
を採っています。これは、和文の太字は線が太いだけで、
横幅が広がるわけではないためです。


\section{NFSSの和文対応の改善}
\pLaTeXe\ \texttt{<2020-04-12>}での修正です。
\LaTeXe\ \texttt{<2020-02-02>}でのNFSSの拡張とは無関係に、
従来から\pLaTeXe{}のNFSSに存在した和文フォントでの不具合に
対処しました。

\paragraph{\cs{fontshape}で和文シェイプが未定義なら維持}
端的には |{\itshape a}| のようなシェイプ変更で
\begin{verbatim}
    Font shape `JT1/mc/m/it' undefined
    using `JT1/mc/m/n' instead on ....
    Font shape `JY1/mc/m/it' undefined
    using `JY1/mc/m/n' instead on ....
\end{verbatim}
という|LaTeX Font Warning|が出るのを抑制することを
目的とした修正です。

\pLaTeXe{}の|\fontshape|は、欧文に加えて和文のシェイプも変更するように
再定義されていました(アスキー\pLaTeXe{}の仕様)。
しかし、これでは|\itshape|のような命令で余分な警告を発してしまいます。
\begin{itemize}
 \item \LaTeXe{}が定義する|\itshape|などのシェイプ変更命令は、
   内部で|\fontshape|を呼び出す。
 \item \pLaTeXe{}の|\fontshape|は、欧文書体だけでなく
   和文書体のシェイプも変更しようとする。
 \item しかし、和文書体のシェイプはほとんど``n''しか用いられず、
   |\DeclareFontShape|での定義も``n''しか与えられないことが多い。
 \item 結果的に、欧文書体のシェイプを変更するつもりでも
   「和文書体のシェイプが未定義」という警告が出てしまう。
\end{itemize}
これを改善するため、\pLaTeXe\ \texttt{<2020-04-12>}では
\begin{itemize}
 \item |\fontshape|が選択しようとした
   和文のシェイプが未定義の場合は、
   和文のシェイプは変更せず、欧文のシェイプのみを変更する
\end{itemize}
という仕様に変更しました。これは|\fontshapeforce|も同様です。

なお、|\kanjishape|や|\kanjishapeforce|は
和文書体のシェイプを変更する意図でしか実行されないため、
未定義かどうかは確認しません。
また、|\usefont|は和欧文両用ですが、一緒に指定された
エンコーディングに応じて|\useroman|と|\usekanji|の一方だけが
実行されることが明白なため、やはり未定義かどうかは確認しません。

\paragraph{sub, ssubの和文対応}
|\DeclareFontShape|で使われるsize functionの
|sub|, |ssub|が和文フォントを正しく取得できない不具合を修正。

これが露呈する具体的な症例としては、GitHubの
|texjporg/jsclasses#53|で報告されている
\begin{itemize}
 \item 日本語の数式ファミリを宣言した状態で
   \file{bm}パッケージを使うとゴシックにならない
\end{itemize}
がありましたが、今回のリリースで解消しました。

\paragraph{\cs{normalfont}末尾の\cs{ignorespaces}削除}
オリジナルの\LaTeXe{}の1995年の修正に追随し、
\pLaTeXe{}の|\normalfont|末尾になぜか残っていた
|\ignorespaces|を削除しました。


\section{その他の修正}
\pLaTeXe\ \texttt{<2020-04-12>}での修正:
\begin{itemize}
 \item |\verb+>+|を和文で挟むと後ろだけアキが入り、前には入らない
   現象への対処(|\texttt{>}|と同様に前後にアキが入るようにした)
\end{itemize}


\section{開発版のテストのお願い}
今後\pLaTeX{}に導入するかもしれない修正パッチや仕様変更のテストに
ご協力ください。\TeX{}ファイルの冒頭(|\documentclass|より前)で
\begin{verbatim}
  \RequirePackage{exppl2e}
\end{verbatim}
と書くことで、現在の開発版をテストすることができます。
現在は
\begin{itemize}
  \item 支柱コマンドで用いられる|\strut|の挙動
  \item 空のフロートだけのページが発生した場合の処理
\end{itemize}
に関するパッチが入っています。

詳細は\file{exppl2e.pdf}を参照してください。ここには、
その他の\pLaTeXe{}の既知の制約事項も記載しています。
\TeX\ ForumやGitHubのIssueでのバグ報告やご意見を歓迎します。
\begin{itemize}
\item \texttt{https://github.com/texjporg/platex}
\item \texttt{https://github.com/texjporg/uplatex}
\end{itemize}

\end{document}
