\documentclass{article}
\begin{document}

{\itshape a}
% 内部で \fontshape が欧文と和文の両方を変えようとする
% => 和文の変更に失敗しても警告は出したくない

{\fontshape{ait}
 \selectfont}
% \fontshape は欧文と和文の両方を変えようとする
% => 和文の変更に失敗しても警告は出したくない

{\fontshapeforce{asl}
 \selectfont}
% \fontshapeforce は欧文と和文の両方を変えようとする
% => 和文の変更に失敗しても警告は出したくない

{\kanjishape{kit}
 \selectfont}
% \kanjishape は和文だけを変更 => 警告すべき

{\kanjishapeforce{ksl}
 \selectfont}
% \kanjishapeforce は和文だけを変更 => 警告すべき

{\usefont{JY1}{mc}{m}{ksc}}
% \usefont は encoding に応じて \useroman と \usekanji の一方だけを実行
% この例は和文横組フォントだけを変えることを意図 => 警告すべき

{\fontshape{asc}
 \selectfont}
% \fontshape は欧文と和文の両方を変えようとする
% => 和文の変更に失敗しても警告は出したくない

{\fontshapeforce{asw}
 \selectfont}
% \fontshapeforce は欧文と和文の両方を変えようとする
% => 和文の変更に失敗しても警告は出したくない

\end{document}
