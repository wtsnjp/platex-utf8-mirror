\documentclass{tarticle}
\usepackage{amsmath}
\setlength{\parindent}{0cm}
\setlength{\textwidth}{8cm}
\begin{document}

% strutbox パッチなしでは数式や数式番号が揃わない

align環境、\verb+&+が1つ %% 少し上へずれ…てはいけない!
\begin{align}
a_1& =b_1+c_1\\
a_2& =b_2+c_2-d_2+e_2
\end{align}

align環境、\verb+&+が3つ %% 少し上へずれ…てはいけない!
\begin{align}
a_{11}& =b_{11}&
  a_{12}& =b_{12}\\
a_{21}& =b_{21}&
  a_{22}& =b_{22}+c_{22}
\end{align}

align環境、\verb+&+が2つ %% 端に付く
\begin{align}
a_{11}& =b_{11}&
  a_{12}\\
a_{21}& =b_{21}&
  a_{22}
\end{align}

align環境、\verb+&+なし %% 端に付く
\begin{align}
a_1=b_1+c_1
\end{align}

\end{document}

%% 以下、参考 %%
「\strutbox の仕様」
https://github.com/texjporg/platex/issues/20
