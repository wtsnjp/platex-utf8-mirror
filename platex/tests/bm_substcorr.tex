% ダメになる例:ゴシックが明朝になる
% disablejfam なので数式内で漢字が使えない最小化
\documentclass[disablejfam]{jarticle}
\usepackage{etoolbox}% for \csshow
% ssub でなく明示定義されていれば大丈夫
%\DeclareFontShape{JY1}{mc}{bx}{n}{<5> <7> <10> goth10}{}
\DeclareSymbolFont{mincho}{JY1}{mc}{m}{n}
\SetSymbolFont{mincho}{bold}{JY1}{mc}{bx}{n}
%\DeclareSymbolFont{mincho}{JT1}{mc}{m}{n}
%\SetSymbolFont{mincho}{bold}{JT1}{mc}{bx}{n}
\usepackage{bm}

\begin{document}

% =======================
{\sffamily\gtfamily ゴシックGT}のはずなのに

\csshow{JY1/gt/m/n/10}%==> \JY1/gt/m/n/10=select font goth10.
\csshow{JT1/gt/m/n/10}%==> \JY1/gt/m/n/10=select font tgoth10.
数式
\setbox0=\hbox{$\empty$}%
が一度でも入ると
% 数式が使われた時に有効だった欧文ファミリに再定義されてしまう
\csshow{JY1/gt/m/n/10}%==> \JY1/gt/m/n/10=select font cmr10. (???)
\csshow{JT1/gt/m/n/10}%==> \JT1/gt/m/n/10=select font cmr10. (???)

{\sffamily\gtfamily 同じサイズは明朝体MCになる}
% =======================

% =======================
{\sffamily\gtfamily\Large 別のサイズは大丈夫GT}

{\Large
だが数式
\setbox0=\hbox{$\empty$}%
が一度でも入ると}

{\sffamily\gtfamily\Large 別のサイズも潰されるGT}
% =======================

\end{document}
