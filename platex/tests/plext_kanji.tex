\documentclass{article}
\usepackage{plext}
\begin{document}
\tableofcontents
\section{テスト}
和文\Kanji{section}和文
English\Kanji{section}English
Num123\Kanji{section}123Num\par
和文 \Kanji{section} 和文
English \Kanji{section} English
Num123 \Kanji{section} 123Num
\section{和文\Kanji{section}和文}
\section{English\Kanji{section}English}
\section{Num123\Kanji{section}123Num}
\section{和文 \Kanji{section} 和文}
\section{English \Kanji{section} English}
\section{Num123 \Kanji{section} 123Num}

\bigskip
%% from abenori
\renewcommand{\theenumi}{\Kanji{enumi}}
\begin{enumerate}
\item ほげ
\begin{enumerate}
\item ふが
\end{enumerate}
\end{enumerate}

\end{document}

% カウンタ引数の後続の数字が漢数字になるのを防ぐ半角スペース
\def\@Kanji#1{\expandafter\kansuji\number #1 }
% より簡潔な定義…こちらを最終的には採用
\def\@Kanji#1{\kansuji #1}

%% 以下、参考 %%
「plext の \@Kanji」
https://github.com/texjporg/platex/issues/33
