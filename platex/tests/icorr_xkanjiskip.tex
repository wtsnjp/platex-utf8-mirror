\documentclass{article}
\begin{document}

% \xkanjiskip が両側に入ることを確認
\xkanjiskip10pt
あ\texttt{abc}い\par
\textbf{いいい}$a$\par

% イタリック補正が入ることを確認(1)
\setbox0=\hbox{\textit{f\textup{af}b}}
\showboxdepth10000
\showboxbreadth10000
\tracingonline1
\showbox0
\box0
\hbox{\textit{f\/\textup{af}b}} % 比較用

% イタリック補正が入ることを確認(2)
\begin{quote}\itshape
You must think of \emph{what} to write
before thinking of \emph{how} to write.
\end{quote}
\begin{quote}\itshape % 比較用
You must think of\/ \emph{what} to write
before thinking of\/ \emph{how} to write.
\end{quote}

% fと漢の間にイタリック補正(kern)と和欧文間空白の両方が入る
string \textit{of}漢字\par
string {\itshape of\/}漢字\par
string {\itshape of}漢字\par % 比較用:これはイタリック補正なし

% 必要なイタリック補正と和欧文間空白が入る
漢\textbf{f}漢f\textbf{漢}f\par
漢\textit{f}漢\par
\textit{f\/\textup{漢}f}\par

\end{document}

%% 以下、参考 %%
「\textbfと直後の欧文とのアキ」
https://oku.edu.mie-u.ac.jp/~okumura/texfaq/qa
55068-55069, 55073-55076, 55081, 55086
「\text... の \xkanjiskip 対策は必要か?」
https://github.com/texjporg/platex/issues/51
