\makeatletter\ifx\documentclass\@twoclasseserror\else
\documentclass[twocolumn]{jarticle}
\fi\makeatother
\begin{document}

% テスト %% &の前後や\\の前に空白なし
\begin{tabular}{|c|c|c|}
(中央)&(中央)&(中央)\\
(中央)&(中央)&(中央)
\end{tabular}

\begin{tabular}{|l|l|l|}
(左揃)&(左揃)&(左揃)\\
(左揃)&(左揃)&(左揃)
\end{tabular}

\begin{tabular}{|r|r|r|}
(右揃)&(右揃)&(右揃)\\
(右揃)&(右揃)&(右揃)
\end{tabular}

% テスト %% &の前後や\\の前に空白あり
\begin{tabular}{|c|c|c|}
(中央) & (中央) & (中央) \\
(中央) & (中央) & (中央)
\end{tabular}

\begin{tabular}{|l|l|l|}
(左揃) & (左揃) & (左揃) \\
(左揃) & (左揃) & (左揃)
\end{tabular}

\begin{tabular}{|r|r|r|}
(右揃) & (右揃) & (右揃) \\
(右揃) & (右揃) & (右揃)
\end{tabular}

% 比較用(\mbox に入れるとグルーが両方とも消える)
% 新コードに変わっても問題が起きないことを確認

% テスト %% &の前後や\\の前に空白なし
\begin{tabular}{|c|c|c|}
\mbox{(中央)}&\mbox{(中央)}&\mbox{(中央)}\\
\mbox{(中央)}&\mbox{(中央)}&\mbox{(中央)}
\end{tabular}

\begin{tabular}{|l|l|l|}
\mbox{(左揃)}&\mbox{(左揃)}&\mbox{(左揃)}\\
\mbox{(左揃)}&\mbox{(左揃)}&\mbox{(左揃)}
\end{tabular}

\begin{tabular}{|r|r|r|}
\mbox{(右揃)}&\mbox{(右揃)}&\mbox{(右揃)}\\
\mbox{(右揃)}&\mbox{(右揃)}&\mbox{(右揃)}
\end{tabular}

% テスト %% &の前後や\\の前に空白あり
\begin{tabular}{|c|c|c|}
\mbox{(中央)} & \mbox{(中央)} & \mbox{(中央)} \\
\mbox{(中央)} & \mbox{(中央)} & \mbox{(中央)}
\end{tabular}

\begin{tabular}{|l|l|l|}
\mbox{(左揃)} & \mbox{(左揃)} & \mbox{(左揃)} \\
\mbox{(左揃)} & \mbox{(左揃)} & \mbox{(左揃)}
\end{tabular}

\begin{tabular}{|r|r|r|}
\mbox{(右揃)} & \mbox{(右揃)} & \mbox{(右揃)} \\
\mbox{(右揃)} & \mbox{(右揃)} & \mbox{(右揃)}
\end{tabular}

\newpage

% テスト %% &の前後や\\の前に空白なし
\begin{tabular}{|c|c|c|}
(長いセル)&(長いセル)&(長いセル)\\
(中央)&(中央)&(中央)
\end{tabular}

\begin{tabular}{|l|l|l|}
(長いセル)&(長いセル)&(長いセル)\\
(左揃)&(左揃)&(左揃)
\end{tabular}

\begin{tabular}{|r|r|r|}
(長いセル)&(長いセル)&(長いセル)\\
(右揃)&(右揃)&(右揃)
\end{tabular}

% テスト %% &の前後や\\の前に空白あり
\begin{tabular}{|c|c|c|}
(長いセル) & (長いセル) & (長いセル) \\
(中央) & (中央) & (中央)
\end{tabular}

\begin{tabular}{|l|l|l|}
(長いセル) & (長いセル) & (長いセル) \\
(左揃) & (左揃) & (左揃)
\end{tabular}

\begin{tabular}{|r|r|r|}
(長いセル) & (長いセル) & (長いセル) \\
(右揃) & (右揃) & (右揃)
\end{tabular}

% 比較用(\mbox に入れるとグルーが両方とも消える)
% 新コードに変わっても問題が起きないことを確認

% テスト %% &の前後や\\の前に空白なし
\begin{tabular}{|c|c|c|}
\mbox{(長いセル)}&\mbox{(長いセル)}&\mbox{(長いセル)}\\
\mbox{(中央)}&\mbox{(中央)}&\mbox{(中央)}
\end{tabular}

\begin{tabular}{|l|l|l|}
\mbox{(長いセル)}&\mbox{(長いセル)}&\mbox{(長いセル)}\\
\mbox{(左揃)}&\mbox{(左揃)}&\mbox{(左揃)}
\end{tabular}

\begin{tabular}{|r|r|r|}
\mbox{(長いセル)}&\mbox{(長いセル)}&\mbox{(長いセル)}\\
\mbox{(右揃)}&\mbox{(右揃)}&\mbox{(右揃)}
\end{tabular}

% テスト %% &の前後や\\の前に空白あり
\begin{tabular}{|c|c|c|}
\mbox{(長いセル)} & \mbox{(長いセル)} & \mbox{(長いセル)} \\
\mbox{(中央)} & \mbox{(中央)} & \mbox{(中央)}
\end{tabular}

\begin{tabular}{|l|l|l|}
\mbox{(長いセル)} & \mbox{(長いセル)} & \mbox{(長いセル)} \\
\mbox{(左揃)} & \mbox{(左揃)} & \mbox{(左揃)}
\end{tabular}

\begin{tabular}{|r|r|r|}
\mbox{(長いセル)} & \mbox{(長いセル)} & \mbox{(長いセル)} \\
\mbox{(右揃)} & \mbox{(右揃)} & \mbox{(右揃)}
\end{tabular}

\newpage

%% p の後に空白があると余分な空行が入る問題

\begin{tabular}{|p{2zw}|p{2zw}|p{2.5zw}|}
あああ) & (あああ & (あああ) \\    % )と\\の間に半角空白がある
いいい)\<&\<(いいい&\<(いいい)\<\\ % 比較用(グルーを明示的に削除)
ううう)&(ううう&(ううう)\\         % 半角空白なし
(かかか & かかか) & (かかか) \\    % )と\\の間に半角空白がある
(ききき\<&\<ききき)&\<(ききき)\<\\ % 比較用(グルーを明示的に削除)
(くくく&くくく)&(くくく)\\         % 半角空白なし
\end{tabular}

\begin{tabular}{|p{2zw}|p{2zw}|p{2zw}|}
あ) & (あ & (あ) \\    % )と\\の間に半角空白がある
い)\<&\<(い&\<(い)\<\\ % 比較用(グルーを明示的に削除)
う)&(う&(う)\\         % 半角空白なし
(か & か) & (か) \\    % )と\\の間に半角空白がある
(き\<&\<き)&\<(き)\<\\ % 比較用(グルーを明示的に削除)
(く&く)&(く)\\         % 半角空白なし
\end{tabular}

% 比較用(\mbox に入れるとグルーが両方とも消える)
% 新コードに変わっても問題が起きないことを確認

\begin{tabular}{|p{2zw}|p{2zw}|p{2zw}|}
\mbox{あ)} & \mbox{(あ} & \mbox{(あ)} \\
\mbox{い)}\<&\<\mbox{(い}&\<\mbox{(い)}\<\\
\mbox{う)}&\mbox{(う}&\mbox{(う)}\\
\mbox{(か} & \mbox{か)} & \mbox{(か)} \\
\mbox{(き}\<&\<\mbox{き)}&\<\mbox{(き)}\<\\
\mbox{(く}&\mbox{く)}&\mbox{(く)}\\
\end{tabular}

\newpage

% セルが空の場合 (forum:2232#p13308)

\begin{tabular}{|r|c|}
& xyz \\
a & 111 \\
bb & 22 \\
ccc & 3
\end{tabular}

\begin{tabular}{|r|c|}
{} & xyz \\
a & 111 \\
bb & 22 \\
ccc & 3
\end{tabular}

\begin{tabular}{|c|c|c|}
 &C&C\\
C& &C\\
C&C&
\end{tabular}

\begin{tabular}{|l|l|l|}
 &L&L\\
L& &L\\
L&L&
\end{tabular}

\begin{tabular}{|r|r|r|}
 &R&R\\
R& &R\\
R&R&
\end{tabular}

\end{document}
