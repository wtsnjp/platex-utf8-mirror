%% <2018-07-28>
\documentclass{plnews}

\publicationyear{2018}% 発行年
\publicationmonth{07}% 発行月
\publicationissue{c11}% 番号
\author{日本語\TeX{}開発コミュニティ(\texttt{https://texjp.org/})}

\def\cs#1{\texttt{\char92 #1}}
\def\pTeX{p\kern-.15em\TeX}
\def\eTeX{$\varepsilon$-\TeX}
\def\epTeX{$\varepsilon$-\pTeX}
\def\pLaTeX{p\kern-.05em\LaTeX}
\def\pLaTeXe{p\kern-.05em\LaTeXe}

\begin{document}

\maketitle

この文書はコミュニティ版\pLaTeXe\ \texttt{<2018-07-28>}について、
\pLaTeXe\ \texttt{<2018-05-20>}からの更新箇所をまとめたものです。


\section{標準クラスの\cs{today}を西暦に}
\pLaTeX{}付属の標準クラス
(jarticle, jbook, jreport, tarticle, tbook, treport)では、
これまで|\today|命令で表示される日付のデフォルトを
\begin{quote}
\和暦\today
\end{quote}
のように元号としていました。
しかし、平成31年(2019年)に改元が予定されていることを機に、
今回のリリースからデフォルトを
\begin{quote}
\西暦\today
\end{quote}
のような西暦に変更しました。
u\pLaTeX{}付属のクラスも同様に変更しています。


\section{シリーズ\texttt{b}も太字に}
\LaTeX{}標準では、太字(|\bfseries|, |\textbf|)を指定すると
{\fontseries{bx}\selectfont bold extended} (|bx|)に切り替わります。

一方、\file{tgtermes}, \file{iwona}などの欧文フォントパッケージを
使うと、太字が{\fontseries{b}\selectfont bold} (|b|)に変わります。

従来の(u)\pLaTeX{}では、和文の太字をシリーズ|bx|だけに割り当て、
|b|には何も設定していませんでした。そのため、後者の場合に
\begin{quote}\scriptsize
\begin{verbatim}
LaTeX Font Warning: Font shape `JY1/mc/b/n' undefined
(Font)              using `JY1/mc/m/n' instead on ...
\end{verbatim}
\end{quote}
のような警告が出て「和文が太字にならない」という挙動でした。
今回からシリーズ|b|にも和文の太字を割り当て、
太字になるようにしました。


\section{アクセント文字の再修正}
\pLaTeXe\ 2016/04/17から2016/07/01にかけて、
「ベースライン補正量がゼロでない場合にアクセント合成文字が乱れる」
というバグを修正しようと試みていました。
この時はトラブルが相次いだため、一旦全てのパッチを撤去したの
ですが(参考:\file{plnewsc03.tex})、今回別の方法で
再度パッチを導入しました。
\begin{itemize}
\item \textbf{デフォルトでは修正パッチは無効}です。
\item |\fixcompositeaccent|命令【新設】\\
      この命令を発行すると、それ以降で\pLaTeX{}用修正パッチが
      有効化されます。グループ内で発行された場合は、
      そのグループ内でのみ修正パッチが有効です。
\item |\nofixcompositeaccent|命令【新設】\\
      |\fixcompositeaccent|の効果を打ち消し、
      元の\LaTeX{}の定義に戻します。
\end{itemize}
|\fixcompositeaccent|命令を発行すると、
「ベースライン補正量がゼロでない場合のアクセントの高さ」
「周囲の和文文字との間に自動挿入される|\xkanjiskip|」などの
挙動がほぼ期待通りになりますが、
一部\LaTeX{}とは異なる挙動(警告・エラー)になる場合があります。
必要に応じて|\fixcompositeaccent|を有効化・無効化することで
対処してください。


\section{開発版のテストのお願い}
バグ報告やご意見、開発版の入手はGitHubへ。
\begin{itemize}
\item \texttt{https://github.com/texjporg/platex}
\item \texttt{https://github.com/texjporg/uplatex}
\end{itemize}

\end{document}
