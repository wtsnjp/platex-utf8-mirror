%% <1999/04/05>
\documentclass{plnews}

\publicationyear{1999}% 発行年
\publicationmonth{4}% 発行月
\publicationissue{5}% 番号
\author{中野 賢(\texttt{<ken-na at ascii.co.jp>})
     \& 富樫 秀昭(\texttt{<hideak-t at ascii.co.jp>})
}

\begin{document}

\maketitle

\section{この文書について}
この文書は、p\LaTeXe{}\texttt{<1999/04/05>}版について、
前回の版(\texttt{<1998/09/01>})からの更新箇所をまとめたものです。
それ以前の変更点については、\textsf{plnews*.tex}やChanges.txtを
参照してください。\LaTeX{}レベルでの更新箇所は、\LaTeX{}に付属の
ltnewsファイルを参照してください。

\section{前バージョンからの修正個所}
\begin{itemize}
\item 和文デフォルトフォントを変更しても、文書の先頭では反映されない
  のを修正した(ありがとう、山本@理科大さん)。
\item \verb|\\|コマンドにオプションを付けた場合、その後ろに余計な
  空白が入ってしまうのを修正した(ありがとう、鈴木@京大さん)。
\item \LaTeX \texttt{<1998/12/01>}に対応した。
\end{itemize}


\section{フォーマットファイル作成時の注意}
現在のp\TeX{}では、8ビットコードの連続を16ビットコードと認識して
しまう場合があります。そのため、フランス語やキリル文字などの
8ビットコードが連続するハイフンパターンはまず使えせん。
例えばcmcyraltパッケージでは、途中でつぎのようなエラーになります。

\begin{verbatim}
(/usr/local/share/texmf/tex/latex/contrib/
other/cmcyralt/rhyphen.tex Russian hyphena
tion
! Bad \patterns.
l.107  . え
           2
?
\end{verbatim}

このときは、``|?|''のプロンプトに対して``|x|''で終了してください。
残念ながら、このハイフンパターンをp\TeX{}で利用することはできません。

そこで、hyphen.cfgを用意して、不用意に他のハイフンパターンを
読み込まないようにしてあります。詳しくはREADME2.txtをご覧ください。

\section{その他}
p\TeX{}やp\LaTeXe{}に関する最新情報は、p\TeX{}ホームページ
\begin{verbatim}
    http://www.ascii.co.jp/pb/ptex
\end{verbatim}
より、入手することができます。

バグ報告やお問い合わせなどは、電子メールで
\begin{verbatim}
    www-ptex@ascii.co.jp
\end{verbatim}
までお願いします。

\end{document}
