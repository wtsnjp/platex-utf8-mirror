% \iffalse meta-comment
%% File: pl209.dtx
%
%  Copyright 1995,1996,1997  ASCII Corporation.
%  Copyright (c) 2010 ASCII MEDIA WORKS
%  Copyright (c) 2016 Japanese TeX Development Community
%
%  This file is part of the pLaTeX2e system (community edition).
%  -------------------------------------------------------------
%
% \fi
%
%
% \setcounter{StandardModuleDepth}{1}
% \StopEventually{}
%
% \iffalse
% \changes{v1.0}{1995/03/28}{Based on latex209.dtx v0.39}
% \changes{v1.0b}{1995/08/30}{Based on latex209.dtx v0.46}
% \changes{v1.0c}{1995/11/21}{Add footnote relatex commands.}
% \changes{v1.0d}{1997/01/17}{Only define for p\LaTeXe relatex codes.}
% \changes{v1.0e}{1997/01/28}{書体変更の二文字コマンドを旧版互換にした。}
% \changes{v1.0f}{1997/06/25}{\cs{em}で和文を強調書体に}
% \fi
%
% \iffalse
%<*package>
\NeedsTeXFormat{pLaTeX2e}
\ProvidesFile{pl209.dtx}[1997/06/25 v1.0f Standard pLaTeX file]
%</package>
%<*driver>
\documentclass{jltxdoc}
\GetFileInfo{pl209.dtx}
\title{p\LaTeXe\\2.09互換モード用マクロ\space\fileversion}
\author{Ken Nakano \& Hideaki Togashi}
\date{作成日:\filedate}
\begin{document}
   \maketitle
   \DocInput{\filename}
\end{document}
%</driver>
% \fi
%
% \section{\dst 用モジュール}
% \dst で以下のモジュール名を指定することで、対象となる部分を取り出す
% ことができます。
%
% \begin{center}
% \begin{tabular}{ll}
% pl209 & \file{pl209.def}ファイルを生成\\
% oldfonts & \file{oldpfont.sty}を生成\\
% style &
%    \begin{tabular}[t]{ll}
%    jarticle & \file{jarticle.sty}ファイルを生成 \\ 
%    jbook    & \file{jbook.sty}ファイルを生成\\
%    jreport  & \file{jreport.sty}ファイルを生成\\
%    tarticle & \file{tarticle.sty}ファイルを生成 \\ 
%    tbook    & \file{tbook.sty}ファイルを生成\\
%    treport  & \file{treport.sty}ファイルを生成
%    \end{tabular}
% \end{tabular}
% \end{center}
%
%
% \section{2.09互換マクロ}
% 2.09用のコマンド定義ファイルがロードされたとき、メッセージを出力します。
% また、\LaTeX{}の2.09コマンドマクロ定義をロードします。
%    \begin{macrocode}
%<*pl209>
\typeout{Entering pLaTeX 2.09 compatibility mode.}
\input{latex209.def}
%</pl209>
%    \end{macrocode}
% フォント選択コマンドのトレースのために\file{ptrace}パッケージをロードします。
% \changes{v1.0e}{1997/02/20}{Typemiss:oldlfont from oldlfonts}
%    \begin{macrocode}
%<oldfonts>\RequirePackage{oldlfont}
%<pl209|oldfonts>\RequirePackage{ptrace}
%    \end{macrocode}
%
% \begin{macro}{\Rensuji}
% \begin{macro}{\prensuji}
% p\LaTeXe{}では、|\Rensuji|, |\prensuji|の動作を|\rensuji|コマンドが
% カバーしています。
%    \begin{macrocode}
%<*pl209>
\let\Rensuji\rensuji
\let\prensuji\rensuji
%</pl209>
%    \end{macrocode}
% \end{macro}
% \end{macro}
%
% \begin{macro}{\@footnotemark}
% \begin{macro}{\@makefnmark}
% 脚注の印を出力するマクロを、組み方向に応じて、脚注の方向が変わるように
% します。
%    \begin{macrocode}
%<*pl209>
\def\@footnotemark{\leavevmode
  \ifhmode\edef\@x@sf{\the\spacefactor}\fi
  \ifydir\@makefnmark
  \else\hbox to\z@{\hskip-.25zw\raise2\cht\@makefnmark\hss}\fi
  \ifhmode\spacefactor\@x@sf\fi\relax}
\def\@makefnmark{\hbox{\ifydir $\m@th^{\@thefnmark}$
  \else\hbox{\yoko$\m@th^{\@thefnmark}$}\fi}}
%</pl209>
%    \end{macrocode}
% \end{macro}
% \end{macro}
%
%    \begin{macrocode}
%<*pl209>
\fontencoding{JY1}
\fontfamily{mc}
\fontsize{10}{15}
%</pl209>
%    \end{macrocode}
%
%    \begin{macrocode}
%<*pl209|oldfonts>
\DeclareSymbolFont{mincho}{JY1}{mc}{m}{n}
\DeclareSymbolFont{gothic}{JY1}{gt}{m}{n}
\DeclareSymbolFontAlphabet\mathmc{mincho}
\DeclareSymbolFontAlphabet\mathgt{gothic}
\SetSymbolFont{mincho}{bold}{JY1}{gt}{m}{n}
\jfam\symmincho
%    \end{macrocode}
% \changes{v1.0e}{1997/01/29}{二文字書体変更コマンドの動作を旧版と同等にした。}
% |\mc|と|\gt|は、和文フォントを変更しますが、欧文フォントには影響しません。
%    \begin{macrocode}
\DeclareRobustCommand\mc{%
    \kanjiencoding{\kanjiencodingdefault}%
    \kanjifamily{\mcdefault}%
    \kanjiseries{\kanjiseriesdefault}%
    \kanjishape{\kanjishapedefault}%
    \selectfont\mathgroup\symmincho}
\DeclareRobustCommand\gt{%
    \kanjiencoding{\kanjiencodingdefault}%
    \kanjifamily{\gtdefault}%
    \kanjiseries{\kanjiseriesdefault}%
    \kanjishape{\kanjishapedefault}%
    \selectfont\mathgroup\symgothic}
%    \end{macrocode}
% |\bf|コマンドは、和文フォントをゴシックにし、欧文フォントをボールドに
% します。
%    \begin{macrocode}
\DeclareRobustCommand\bf{\normalfont\bfseries\mathgroup\symbold\jfam\symgothic}
%    \end{macrocode}
% |\rm|, |\sf|, |\sl|, |\sc|, |\it|, |\tt|の各コマンドを、欧文ファミリだけを
% デフォルトフォントから属性を変更するようにし、和文フォントは影響を
% 受けないように修正します。
%    \begin{macrocode}
\DeclareRobustCommand\roman@normal{%
    \romanencoding{\encodingdefault}%
    \romanfamily{\familydefault}%
    \romanseries{\seriesdefault}%
    \romanshape{\shapedefault}%
    \selectfont\ignorespaces}
\DeclareRobustCommand\rm{\roman@normal\rmfamily\mathgroup\symoperators}
\DeclareRobustCommand\sf{\roman@normal\sffamily\mathgroup\symsans}
\DeclareRobustCommand\sl{\roman@normal\slshape\mathgroup\symslanted}
\DeclareRobustCommand\sc{\roman@normal\scshape\mathgroup\symsmallcaps}
\DeclareRobustCommand\it{\roman@normal\itshape\mathgroup\symitalic}
\DeclareRobustCommand\tt{\roman@normal\ttfamily\mathgroup\symtypewriter}
%    \end{macrocode}
%
% \begin{macro}{\em}
% \changes{v1.0f}{1997/06/25}{\cs{em}で和文を強調書体に}
% |\em|コマンドで、和文フォントも|\gt|に切り替えるようにしました。
%    \begin{macrocode}
\DeclareRobustCommand\em{%
  \@nomath\em
  \ifdim \fontdimen\@ne\font>\z@\mc\rm\else\gt\it\fi}
%</pl209|oldfonts>
%    \end{macrocode}
% \end{macro}
%
%    \begin{macrocode}
%<*pl209>
\let\mcfam\symmincho
\let\gtfam\symgothic
\renewcommand\vpt   {\edef\f@size{\@vpt}\rm\mc}
\renewcommand\vipt  {\edef\f@size{\@vipt}\rm\mc}
\renewcommand\viipt {\edef\f@size{\@viipt}\rm\mc}
\renewcommand\viiipt{\edef\f@size{\@viiipt}\rm\mc}
\renewcommand\ixpt  {\edef\f@size{\@ixpt}\rm\mc}
\renewcommand\xpt   {\edef\f@size{\@xpt}\rm\mc}
\renewcommand\xipt  {\edef\f@size{\@xipt}\rm\mc}
\renewcommand\xiipt {\edef\f@size{\@xiipt}\rm\mc}
\renewcommand\xivpt {\edef\f@size{\@xivpt}\rm\mc}
\renewcommand\xviipt{\edef\f@size{\@xviipt}\rm\mc}
\renewcommand\xxpt  {\edef\f@size{\@xxpt}\rm\mc}
\renewcommand\xxvpt {\edef\f@size{\@xxvpt}\rm\mc}
%</pl209>
%    \end{macrocode}
% そして、最後に\file{pl209.cfg}というファイルがあれば、それをロードします。
%    \begin{macrocode}
%<pl209>\InputIfFileExists{pl209.cfg}{}{}
%    \end{macrocode}
%
%
% \section{スタイルファイル}
% 以下は、p\LaTeX~2.09での標準スタイルファイルです。
% p\LaTeXe{}のクラスファイルをロードするようにしています。
%    \begin{macrocode}
%<*style>
%<*jarticle|jbook|jreport|tarticle|tbook|treport>
\NeedsTeXFormat{pLaTeX2e}
%</jarticle|jbook|jreport|tarticle|tbook|treport>
%<*jarticle>
\@obsoletefile{jarticle.cls}{jarticle.sty}
\LoadClass{jarticle}
%</jarticle>
%<*tarticle>
\@obsoletefile{tarticle.cls}{tarticle.sty}
\LoadClass{tarticle}
%</tarticle>
%<*jbook>
\@obsoletefile{jbook.cls}{jbook.sty}
\LoadClass{jbook}
%</jbook>
%<*tbook>
\@obsoletefile{tbook.cls}{tbook.sty}
\LoadClass{tbook}
%</tbook>
%<*jreport>
\@obsoletefile{jreport.cls}{jreport.sty}
\LoadClass{jreport}
%</jreport>
%<*treport>
\@obsoletefile{treport.cls}{treport.sty}
\LoadClass{treport}
%</treport>
%</style>
%    \end{macrocode}
%
% \Finale
%
\endinput
