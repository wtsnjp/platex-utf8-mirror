%% <2020-10-01>
\documentclass{plnews}

\publicationyear{2020}% 発行年
\publicationmonth{10}% 発行月
\publicationissue{c15}% 番号
\author{日本語\TeX{}開発コミュニティ(\texttt{https://texjp.org/})}

\def\cs#1{\texttt{\char92\nobreak #1}}
\def\pTeX{p\kern-.15em\TeX}
\def\eTeX{$\varepsilon$-\TeX}
\def\epTeX{$\varepsilon$-\pTeX}
\def\pLaTeX{p\kern-.05em\LaTeX}
\def\pLaTeXe{p\kern-.05em\LaTeXe}
\xspcode`\\=1

\begin{document}

\maketitle

コミュニティ版\pLaTeXe\ \texttt{<2020-10-01>}について、
\pLaTeXe\ \texttt{<2020-04-12>}からの更新箇所をまとめます。


\section{\LaTeXe\ \texttt{<2020-10-01>}対応}
→参考:|texjporg/platex#94|

新しい\LaTeXe\ \texttt{<2020-10-01>}では、
フックの方式(A hook management system)が一般化されました
(\file{ltnews32}, \file{lthooks-doc}も参照)。
古くから個別に定義されていたフック(|\@begindocumenthook|等)や
\LaTeXe\ \texttt{<2020-02-02>}で追加されたNFSSのフック
(|\@defaultfamilyhook|, |\@sffamilyhook|等)も再構成されたため、
対応を施しました。
% [TODO] まだ plfonts.dtx は調整していない。

さらに、ページ出力に関わるコマンドも一新されました
(\file{ltnews32}, \file{ltshipout-doc}も参照)。特に
\begin{itemize}
\item |\AtBeginDvi|の定義変更
\item |atbegshi|/|everyshi|相当の機能のカーネル化
\end{itemize}
が\pLaTeXe{}にも影響しますので、対応を施しました。
これにより、|pxatbegshi|/|pxeveryshi|パッケージに相当する
機能(|plautopatch|パッケージも参照)が\pLaTeXe{}カーネルに
取り込まれたことになります。

注意:実装上の都合により、縦組クラスでは
「|\AtBeginShipout|の中身が外部垂直モードで実行されること」を
想定した使用は\emph{サポートされません}。
(例:|aminophen/platex-tools#15|)


\section{バグ修正}
% [TODO] 準備中。


\section{開発版のテストのお願い}
今後\pLaTeX{}に導入するかもしれない修正パッチや仕様変更のテストに
ご協力ください。\TeX{}ファイルの冒頭(|\documentclass|より前)で
\begin{verbatim}
  \RequirePackage{exppl2e}
\end{verbatim}
と書くことで、現在の開発版をテストすることができます。
詳細は\file{exppl2e.pdf}を参照してください。ここには、
その他の\pLaTeXe{}の既知の制約事項も記載しています。
\TeX\ ForumやGitHubのIssueでのバグ報告やご意見を歓迎します。
\begin{itemize}
\item \texttt{https://github.com/texjporg/platex}
\item \texttt{https://github.com/texjporg/uplatex}
\end{itemize}

\end{document}
