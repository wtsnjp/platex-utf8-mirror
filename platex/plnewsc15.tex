%% <2020-10-01>
\documentclass{plnews}

\publicationyear{2020}% 発行年
\publicationmonth{10}% 発行月
\publicationissue{c15}% 番号
\author{日本語\TeX{}開発コミュニティ(\texttt{https://texjp.org/})}

\def\cs#1{\texttt{\char92\nobreak #1}}
\def\pTeX{p\kern-.15em\TeX}
\def\eTeX{$\varepsilon$-\TeX}
\def\epTeX{$\varepsilon$-\pTeX}
\def\pLaTeX{p\kern-.05em\LaTeX}
\def\pLaTeXe{p\kern-.05em\LaTeXe}
\xspcode`\\=1

\begin{document}

\maketitle

コミュニティ版\pLaTeXe\ \texttt{<2020-10-01>}について、
\pLaTeXe\ \texttt{<2020-04-12>}からの更新箇所を
まとめました。u\pLaTeXe{}も同時に更新してください。


\section{\LaTeXe\ \texttt{<2020-10-01>}対応}
→参考:|texjporg/platex#94|

新しい\LaTeXe\ \texttt{<2020-10-01>}では、
フックの方式(A hook management system)が一般化されました
(\file{ltnews32}, \file{lthooks-doc}も参照)。
古くから個別に定義されていたフック(|\@begindocumenthook|等)や
\LaTeXe\ \texttt{<2020-02-02>}で追加されたNFSSのフック
(|\@defaultfamilyhook|, |\@sffamilyhook|等)も再構成されたため、
対応を施しました。

さらに、ページ出力に関わるコマンドも一新されました
(\file{ltnews32}, \file{ltshipout-doc}も参照)。特に
\begin{itemize}
\item |\AtBeginDvi|の定義変更
\item |atbegshi|/|everyshi|相当の機能のカーネル化
\end{itemize}
が\pLaTeXe{}にも影響しますので、対応を施しました。
これにより、|pxatbegshi|/|pxeveryshi|パッケージに相当する
機能(|plautopatch|パッケージも参照)が\pLaTeXe{}カーネルに
取り込まれたことになります。

注意:実装上の都合により、縦組クラスでは
「|\AtBeginShipout|の中身が外部垂直モードで実行されること」を
想定した使用は\emph{サポートされません}。
(例:|aminophen/platex-tools#15|)


\section{新NFSSの追加修正}
前回の\LaTeXe\ \texttt{<2020-02-02>}で大幅に拡張された
NFSS(フォント選択の仕組み)の新機能について、
\LaTeXe\ \texttt{<2020-10-01>}で追加修正が入りましたので
追随しました。(参考:|latex3/latex2e#315|)

具体的には「|\DeclareFontSeriesDefault|の指定が
|\normalfont|に反映されない問題」への修正です。
再現例は以下を参照してください。
\begin{verbatim}
  % roman-default = roman-medium
  % 明朝のデフォルト=明朝の中字
  \documentclass{article}
  % roman-medium-default -> roman-bold
  % 明朝の中字のデフォルト→明朝の太字へ
  %   ※明朝の太字はゴシックの中字に置換される
  \DeclareFontSeriesDefault[rm]{md}{b}
  \DeclareFontSeriesDefault[mc]{md}{b}
  \begin{document}
  roman-default is bold
  明朝のデフォルトは太字になった

  \normalfont % 2020-02-02: 太字にならなかった
  roman-default is bold?
  明朝のデフォルトは太字になったか?

  \mdseries
  roman-medium is bold
  明朝のミディアムは太字になった
  \end{document}
\end{verbatim}


\section{expl3文法の\pLaTeX{}版}
\LaTeXe\ \texttt{<2020-02-02>}でexpl3がフォーマットに
組み込まれたことを受け、
\pLaTeX{}でも\pTeX{}系列の追加機能をexpl3文法に則って
利用しやすくするため、本リリースで新設しました。
\pLaTeXe\ \texttt{<2020-10-01>}では、この機能を
フォーマットに組み込んであります。

現時点では、組方向変更(|\tate|等)と組方向判定(|\iftdir|等)を
ラップするコマンドを用意してあります。詳細は
\file{pldoc.pdf}の``plexpl3.dtx''の節を参照してください。


\section{バグ修正}
\begin{itemize}
 \item \LaTeXe~2017/01/01以降で空のフロートだけのページが
  発生した場合、縦組クラスではフッタの位置が持ち上がっていたので
  修正しました。(|#78|)
 \item 禁則パラメータ設定ファイル(kinsoku.tex)に
  |\inhibitxspcode`!=1|の設定が抜けていたので追加しました。
  (|ptex-base#8|)
\end{itemize}


\section{開発版のテストのお願い}
今後\pLaTeX{}に導入するかもしれない修正パッチや仕様変更のテストに
ご協力ください。\TeX{}ファイルの冒頭(|\documentclass|より前)で
\begin{verbatim}
  \RequirePackage{exppl2e}
\end{verbatim}
と書くことで、現在の開発版をテストすることができます。
詳細は\file{exppl2e.pdf}を参照してください。ここには、
その他の\pLaTeXe{}の既知の制約事項も記載しています。
\TeX\ ForumやGitHubのIssueでのバグ報告やご意見を歓迎します。
\begin{itemize}
\item \texttt{https://github.com/texjporg/platex}
\item \texttt{https://github.com/texjporg/uplatex}
\end{itemize}

\end{document}
