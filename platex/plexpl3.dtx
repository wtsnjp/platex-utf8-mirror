% \iffalse meta-comment
%% File: plexpl3.dtx
%
%  Copyright (c) 2020 Japanese TeX Development Community
%
%  This file is part of the pLaTeX2e system (community edition).
%  -------------------------------------------------------------
%
% \fi
%
%
% \iffalse
% \changes{v1.0}{2020/09/26}{初版:p\TeX{}の条件文を定義}
% \fi
%
% \iffalse
%<*driver>
\NeedsTeXFormat{pLaTeX2e}
\ProvidesFile{plexpl3.dtx}
\documentclass{jltxdoc}
\GetFileInfo{plexpl3.dtx}
\author{Japanese \TeX\ Development Community}
\title{The \textsf{plexpl3} package}
\date{作成日:\filedate}
\begin{document}
  \newcommand\Lpack[1]{\textsf{#1}}
  \maketitle
  \DocInput{\filename}
\end{document}
%</driver>
% \fi
%
% \LaTeX3 (expl3)で用意されていない「p\TeX{}系列の独自機能」を
% expl3の文法で使えるようにするコードです。
%
% \setcounter{StandardModuleDepth}{1}
% \StopEventually{}
%
% \section{コード}
%
% パッケージを宣言します。
%    \begin{macrocode}
%<*package>
\NeedsTeXFormat{pLaTeX2e}
\RequirePackage{expl3}
\ProvidesExplPackage{plexpl3}{2020-09-26}{1.0}
  {pTeX/upTeX-specific additions to expl3}
%</package>
%    \end{macrocode}
%
%    \begin{macrocode}
%<plcore>\ifx\ExplSyntaxOn\@undefined\else
%<plcore>\ExplSyntaxOn
%<*plcore|package>
\input plexpl3.code.tex
%</plcore|package>
%<plcore>\ExplSyntaxOff
%<plcore>\fi
%    \end{macrocode}
%
% \section{pTeX{}系列の条件文}
%
% p\TeX{}系列の条件文を|\if-|トークンに見えるようにします。
% \changes{v1.0}{2020/09/26}{初版:p\TeX{}の条件文を定義}
%    \begin{macrocode}
%<*code>
\if_cs_exist:N \__platex_explLoaded:
  \GenericInfo{}
    {Skipping:~ plexpl3~ code~ already~ part~ of~ the~ format}%
  \exp_after:wN \endinput
\fi:
\cs_new:Npn \__platex_explLoaded: {  }
% additions to l3box.dtx (pTeX/upTeX-specific)
\cs_new_eq:NN \if_box_tate:N \tex_iftbox:D
\cs_new_eq:NN \if_box_yoko:N \tex_ifybox:D
\cs_new_eq:NN \if_box_dtou:N \tex_ifdbox:D
\cs_new_eq:NN \if_box_math:N \tex_ifmbox:D
% writing directions (pTeX/upTeX-specific)
\cs_new_eq:NN \if_direction_tate: \tex_iftdir:D
\cs_new_eq:NN \if_direction_yoko: \tex_ifydir:D
\cs_new_eq:NN \if_direction_dtou: \tex_ifddir:D
\cs_new_eq:NN \if_direction_math: \tex_ifmdir:D
%</code>
%    \end{macrocode}
%
% \Finale
%
\endinput
