%% <2000/11/03>
\documentclass{plnews}

\publicationyear{2000}% 発行年
\publicationmonth{11}% 発行月
\publicationissue{6}% 番号
\author{中野 賢(\texttt{<ken-na at ascii.co.jp>})
     \& 富樫 秀昭(\texttt{<hideak-t at ascii.co.jp>})
}

\begin{document}

\maketitle

\section{この文書について}
この文書は、p\LaTeXe{}\texttt{<2000/11/03>}版について、
前回の版(\texttt{<1999/08/09>})からの更新箇所をまとめたものです。
それ以前の変更点については、\textsf{plnews*.tex}やChanges.txtを
参照してください。\LaTeX{}レベルでの更新箇所は、\LaTeX{}に付属の
ltnewsファイルを参照してください。

\section{前バージョンからの主な修正個所}
\begin{itemize}
\item 配布形態をte\TeX{}ライブラリの形式に変更した。
\item |nidanfloat|パッケージを付け加えた。
\item |\text..|コマンドの左側に|\xkanjiskip|が入らないのを修正(ありがと
う、乙部@東大さん)
\item tarticle, tbook, treportで、文頭の全角開き括弧類が下がる現象に対処。
|\adjustbaseline|を修正しました。
\item \LaTeX \texttt{<2000/06/01>}に対応した。
\end{itemize}


\section{te\TeX{}ライブラリ形式での配布}
\textit{\TeX{} Live}というTUGで配布している\TeX{}システムを集めた
CD--ROMがあり、TUGboat購読者にはこれがTUGboatと一緒に定期的に配布されて
います。te\TeX{}(Thomas Esserによる)は\textit{\TeX{} Live}用に集められ
た\TeX{}のことです。

te\TeX{}のTDS(\textit{\TeX{} Directory Structure})に従った配布物には、
ポーランド語の\LaTeX{}用に|platex|というディレクトリが含まれており、
p\TeX{}のplatexと重なります。この問題を避けるために、p\TeX{}用のディレク
トリを|texmf|直下に|texmf/ptex|と作り、p\TeX{}ではそちらを優先して使うよ
うにし、te\TeX{}ライブラリに合わせた形でp\TeX{}関連のライブラリをまとめ
て|ptex-texmf-*.tar.gz|として配布しています。

このディレクトリ構成は従来のディレクトリ構成と異なっており、\LaTeXe{}の
|*.ins|にはディレクトリ名を記述するので、te\TeX{}用の配布物とは別に、こ
れまでのようにp\LaTeXe{}のパッケージを作ると、ディレクトリ名の記述だけが
異なり、他は全く同じ2種類の配布物が出来上がることになります。この状態は
望ましくないので、\TeX{}の世界全体がte\TeX{}にシフトしてきていることも考
慮し\footnote{例えば、オリジナルのdvipsの最新版はte\TeX{}に含まれるもの
だけとなっています。}、p\LaTeXe{}の配布もte\TeX{}ライブラリ形式での配布
形態に絞ることにしました。今後p\LaTeXe{}のバージョンアップは、
|ptex-texmf*.tar.gz|アーカイブに含まれる形で行なうことになります。


\section{nidanfloatパッケージの使い方}
|nidanfloat|パッケージは、二段組時に段抜きのフロートをページ下部にも配置
できるようにするためのパッケージです。通常は、以下のような使い方になるで
しょう。ページ下部に1段の幅に収まらない|filename.eps|を出力する場合です。
\begin{verbatim}
\documentclass[twocolumn]{jarticle}
\usepackage{graphics}
\usepackage{nidanfloat}
\begin{document}
     <本文>
\begin{figure*}[b]
\includegraphics{filename.eps}
\caption{キャプション}
\end{figure*}
     <本文>
\end{document}
\end{verbatim}
このように、二段組で|\usepackage{nidanfloat}|をプリアンブルに指定して、
|figure|環境のオプションで|b|を指定します。オプションの意味は、通常の
|figure|環境と同じです。|figure|環境のオプションを指定しない場合は、デフォ
ルトで|tb|が指定されたものと見なされます。

その他、追加されたパラメータなどに付いては、|nidanfloat.dtx|をご覧くださ
い。


\section{tarticle, tbook, treportで、文頭の全角開き括弧類が下がる問題}
tarticle, tbook, treportで、文頭の全角開き括弧類が下がるという現象のご指
摘を頂きました。このアキは、|\adustbaseline|で出力されていたものです。具
体的には、|\tbaselineshift|に2度続けて値を指定すると、その後にある全角開
き括弧類の前に余分なアキが出力されるようです。|\adustbaseline|では、縦組
のベースライン位置を補正する際に|\tbaselineshift|を初期化し、その後に計
算値を設定するということをしていたために、その直後に全角開き括弧類がくる
と余分なアキが出力されていたものです。|\tbaselineshift|への連続した値の
設定を行なわなければこの問題は起きないので、このバージョンで
|\adustbaseline|の最初で行なっていた|\tbaselineshift|の初期化を行なわな
いように変更しました。


\section{\LaTeX \texttt{<2000/06/01>}に対応}
\LaTeX{}のバージョンアップが今回から1年毎になりましたので、p\LaTeXe{}の
更新も基本的に今後は\LaTeX{}に合わせて1年毎になります。


\section{フォーマットファイル作成時の注意}
現在のp\TeX{}では、8ビットコードの連続を16ビットコードと認識して
しまう場合があります。そのため、フランス語やキリル文字などの
8ビットコードが連続するハイフンパターンはまず使えせん。
例えばcmcyraltパッケージでは、途中でつぎのようなエラーになります。

\begin{verbatim}
(/usr/local/share/texmf/tex/latex/contrib/
other/cmcyralt/rhyphen.tex Russian hyphena
tion
! Bad \patterns.
l.107  . え
           2
?
\end{verbatim}

このときは、``|?|''のプロンプトに対して``|x|''で終了してください。
残念ながら、このハイフンパターンをp\TeX{}で利用することはできません。

そこで、hyphen.cfgを用意して、不用意に他のハイフンパターンを
読み込まないようにしてあります。詳しくはREADME2.txtをご覧ください。

\section{その他}
p\TeX{}やp\LaTeXe{}に関する最新情報は、p\TeX{}ホームページ
\begin{verbatim}
    http://www.ascii.co.jp/pb/ptex
\end{verbatim}
より、入手することができます。

バグ報告やお問い合わせなどは、電子メールで
\begin{verbatim}
    www-ptex@ascii.co.jp
\end{verbatim}
までお願いします。

\end{document}
