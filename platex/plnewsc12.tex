%% <2018-12-01>
\documentclass{plnews}

\publicationyear{2018}% 発行年
\publicationmonth{12}% 発行月
\publicationissue{c12}% 番号
\author{日本語\TeX{}開発コミュニティ(\texttt{https://texjp.org/})}

\def\cs#1{\texttt{\char92 #1}}
\def\pTeX{p\kern-.15em\TeX}
\def\eTeX{$\varepsilon$-\TeX}
\def\epTeX{$\varepsilon$-\pTeX}
\def\pLaTeX{p\kern-.05em\LaTeX}
\def\pLaTeXe{p\kern-.05em\LaTeXe}

\begin{document}

\maketitle

この文書はコミュニティ版\pLaTeXe\ \texttt{<2018-12-01>}について、
\pLaTeXe\ \texttt{<2018-07-28>}からの更新箇所をまとめたものです。


\section{\file{plext}のバグ修正}
\file{plext}パッケージが拡張する|\pcaption|命令について、
下記の仕様変更とバグ修正を行いました。
\begin{description}
\item[仕様変更]
  本文が縦組の時、キャプションも縦組にする。
\item[バグ修正]
  本文が縦組で、キャプションを横組にする場合に
  「どんなに短いキャプションでも、幅を増やして
  一行に収めることができなかった」というバグの修正。
\end{description}
詳細はGitHub issue 76及び\TeX\ forum 2506を参照してください。


\section{\LaTeXe\ \texttt{<2018-12-01>}対応}
\LaTeXe\ \texttt{<2018-12-01>}では、
外部ファイルへの書き出しと読み込みを要する処理(目次など)で
「行末の空白文字に由来する不自然な空白が入ることがある」という
問題が修正される予定です。
このため、(u)\pLaTeX{}の標準クラスもこの修正に追随しました。
詳細はGitHub issue 79を参照してください。


\section{開発版のテストのお願い}
今後p\LaTeX{}に導入するかもしれない修正パッチや仕様変更のテストに
ご協力ください。\TeX{}ファイルの冒頭(|\documentclass|より前)で
\begin{verbatim}
\RequirePackage{exppl2e}
\end{verbatim}
と書くことで、現在の開発版をテストすることができます。
現在は、支柱コマンドで用いられる|\strut|の挙動に関するパッチが
入っています。

詳細は\file{exppl2e.pdf}を参照してください。
\TeX\ ForumやGitHubのIssueでのバグ報告やご意見を歓迎します。
\begin{itemize}
\item \texttt{https://github.com/texjporg/platex}
\item \texttt{https://github.com/texjporg/uplatex}
\end{itemize}

\end{document}
