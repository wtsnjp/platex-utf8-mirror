%% <2019-04-06>
\documentclass{plnews}

\publicationyear{2019}% 発行年
\publicationmonth{04}% 発行月
\publicationissue{c13}% 番号
\author{日本語\TeX{}開発コミュニティ(\texttt{https://texjp.org/})}

\def\cs#1{\texttt{\char92 #1}}
\def\pTeX{p\kern-.15em\TeX}
\def\eTeX{$\varepsilon$-\TeX}
\def\epTeX{$\varepsilon$-\pTeX}
\def\pLaTeX{p\kern-.05em\LaTeX}
\def\pLaTeXe{p\kern-.05em\LaTeXe}

\begin{document}

\maketitle

この文書はコミュニティ版\pLaTeXe\ \texttt{<2019-04-06>}について、
\pLaTeXe\ \texttt{<2018-12-01>}からの更新箇所をまとめたものです。


\section{標準クラスの新元号対応}
\pLaTeX{}付属の標準クラス
(jarticle, jbook, jreport, tarticle, tbook, treport)で
\verb+\和暦+を指定した場合の\verb+\today+について、
2019年5月1日から施行予定の新元号「令和」に対応しました。
{\year=2019 \month=4 \day=30 \西暦\today は\和暦\today}、
{\year=2019 \month=5 \day=1 \西暦\today は\和暦\today}、
{\year=2020 \month=1 \day=2 \西暦\today は\和暦\today}と
なります。

また、縦数式ディレクションで\verb+\today+を使った場合、
従来は縦組と同じく漢数字に変換されていましたが、
このバージョンからは算用数字としました。


\section{\LaTeXe\ \texttt{<2019-04-01>}対応}
\LaTeXe\ \texttt{<2019-04-01>}(予定)の表組み(tabular環境)の
マクロへの微修正に追随しました。


\section{開発版のテストのお願い}
今後\pLaTeX{}に導入するかもしれない修正パッチや仕様変更のテストに
ご協力ください。\TeX{}ファイルの冒頭(|\documentclass|より前)で
\begin{verbatim}
\RequirePackage{exppl2e}
\end{verbatim}
と書くことで、現在の開発版をテストすることができます。
現在は、支柱コマンドで用いられる|\strut|の挙動に関するパッチ、
空のフロートだけのページが発生した場合の処理に関するパッチが
入っています。

詳細は\file{exppl2e.pdf}を参照してください。ここには、
その他の\pLaTeXe{}の既知の制約事項も記載しています。
\TeX\ ForumやGitHubのIssueでのバグ報告やご意見を歓迎します。
\begin{itemize}
\item \texttt{https://github.com/texjporg/platex}
\item \texttt{https://github.com/texjporg/uplatex}
\end{itemize}

\end{document}
