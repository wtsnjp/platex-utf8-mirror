%% <2019-04-06> and <2019-10-01>
\documentclass{plnews}

\publicationyear{2019}% 発行年
\publicationmonth{10}% 発行月
\publicationissue{c13}% 番号
\author{日本語\TeX{}開発コミュニティ(\texttt{https://texjp.org/})}

\def\cs#1{\texttt{\char92\nobreak #1}}
\def\pTeX{p\kern-.15em\TeX}
\def\eTeX{$\varepsilon$-\TeX}
\def\epTeX{$\varepsilon$-\pTeX}
\def\pLaTeX{p\kern-.05em\LaTeX}
\def\pLaTeXe{p\kern-.05em\LaTeXe}
\xspcode`\\=1

\begin{document}

\maketitle

コミュニティ版\pLaTeXe\ \texttt{<2019-04-06>}および
\texttt{<2019-10-01>}について、
\pLaTeXe\ \texttt{<2018-12-01>}からの更新箇所をまとめます。


\section{標準クラスの新元号対応}
\pLaTeXe\ \texttt{<2019-04-06>}では、付属の標準クラス
(jarticle, jbook, jreport, tarticle, tbook, treport)で
\cs{和暦} を指定した場合の \cs{today} について、
2019年5月1日から施行予定の新元号「令和」に対応しました。
\begin{itemize}
\item {\year=2019 \month=4 \day=30
       \西暦\today $\rightarrow$ \和暦\today}
\item {\year=2019 \month=5 \day=1
       \西暦\today $\rightarrow$ \和暦\today}
\item {\year=2020 \month=1 \day=2
       \西暦\today $\rightarrow$ \和暦\today}
\end{itemize}
また、縦数式ディレクションで \cs{today} を使った場合、
従来は縦組と同じく漢数字に変換されていましたが、
このバージョンからは算用数字としました。


\section{\LaTeXe\ \texttt{<2019-10-01>}対応}
オリジナルの\LaTeX に合わせるため、\pLaTeX へも
いくつか修正を加えました。

\pLaTeXe\ \texttt{<2019-04-06>}時点では
\begin{itemize}
\item 表組み(tabular環境)のマクロの微修正
\end{itemize}
に対応しました。

\pLaTeXe\ \texttt{<2019-10-01>}時点では、さらに
\begin{itemize}
\item \cs{DeclareErrorKanjiFont}が\cs{k@family}等を定義しない
  ように(\cs{DeclareErrorFont}に追随)
\item ユーザ用コマンドをrobustに
  \begin{itemize}
    \item \cs{strut}関係 (\cs{\{t,z,y\}strut})
    \item \cs{usefont}関係 (\cs{use\{roman,kanji\}})
    \item \cs{userelfont}, \cs{adjustbaseline}
    \item \cs{AtBeginDvi}, \cs{underline}
    \item plextパッケージの\cs{bou}, \cs{kasen}
  \end{itemize}
\end{itemize}
にも対応しました。

\section{開発版のテストのお願い}
今後\pLaTeX{}に導入するかもしれない修正パッチや仕様変更のテストに
ご協力ください。\TeX{}ファイルの冒頭(|\documentclass|より前)で
\begin{verbatim}
  \RequirePackage{exppl2e}
\end{verbatim}
と書くことで、現在の開発版をテストすることができます。
現在は
\begin{itemize}
  \item 支柱コマンドで用いられる|\strut|の挙動
  \item 空のフロートだけのページが発生した場合の処理
\end{itemize}
に関するパッチが入っています。

詳細は\file{exppl2e.pdf}を参照してください。ここには、
その他の\pLaTeXe{}の既知の制約事項も記載しています。
\TeX\ ForumやGitHubのIssueでのバグ報告やご意見を歓迎します。
\begin{itemize}
\item \texttt{https://github.com/texjporg/platex}
\item \texttt{https://github.com/texjporg/uplatex}
\end{itemize}

\end{document}
