%% <2016/06/10>
\documentclass{plnews}

\publicationyear{2016}% 発行年
\publicationmonth{06}% 発行月
\publicationissue{c2}% 番号
\author{日本語\TeX{}開発コミュニティ(\texttt{https://texjp.org/})}

\def\pTeX{p\kern-.15em\TeX}
\def\epTeX{$\varepsilon$-\pTeX}
\def\pLaTeX{p\kern-.05em\LaTeX}
\def\pLaTeXe{p\kern-.05em\LaTeXe}

\begin{document}

\maketitle

\section{この文書について}
この文書はコミュニティ版\pLaTeXe\ \texttt{<2016/06/10>}について、
\pLaTeXe\ \texttt{<2016/05/07>}からの更新箇所をまとめたものです。
以前のアスキー版の変更点については、
\file{plnews*.tex}や\file{Changes\_asciimw.txt}を参照してください。
コミュニティ版の変更点については、\file{plnewsc*.tex}を参照してください。
\LaTeX{}レベルでの更新箇所は、\LaTeX{}に付属の\file{ltnews*.tex}などを
参照してください。


\section{アクセント文字のバグ修正}
\pLaTeX\ \texttt{<2016/05/07>}で「縦組で「\AA{}」が乱れるバグの修正」を
導入しましたが、この変更で「すべての合成文字でリガチャやカーニングがきか
ない、周囲に|\xkanjiskip|が入らない、\file{ucs}パッケージが使えない」
などという不具合が入ってしまいました。この問題を修正しました (Issue \#5) 。


\section{8-bitフォントエンコーディングの欧文文字周囲のスペース}
アスキー版\pLaTeX{}が作られた頃に比べて、最近はT1エンコーディングなどの
8-bitフォントエンコーディングが多く用いられるようになりました。128--255の
文字は欧文文字ですので、新しいフォーマットでは周囲に|\xkanjiskip|が入る
ように\file{kinsoku.tex}で|\xspcode=3|に設定しました。これは、奥村さん
の\file{jsclasses}や田中さんのu\pLaTeX{}と同等の対処です (Issue \#6) 。


\section{\file{pfltrace}パッケージの追加}
\LaTeX\ \texttt{<2014/05/01>}で、\file{fltrace}パッケージが追加されまし
た。これは\LaTeX{}カーネルのソースに隠れていたフロート配置アルゴリズムの
トレースに使うコードをパッケージの形に抽出したもの\footnote{参考:
\LaTeX\ News Issue 21、\file{ltnews21.tex}}で、実は\pLaTeX{}カーネルの
ソースにも同様に隠れていたコードがありました。これを取り出したものが
\file{pfltrace}パッケージです。

フロートアルゴリズムの動作を確認したい場合は
\begin{verbatim}
  \usepackage{pfltrace} \tracefloats
\end{verbatim}
のように書きます。トレースを中断するには|\tracefloatsoff|を使い、
現在のさまざまなフロートパラメータの値を確認するには|\tracefloatvals|を
使います。実際の処理は\file{fltrace}パッケージを読み込むことで行い、
\pLaTeX{}特有の変更のみが\file{pfltrace}パッケージに書かれています
\footnote{フォント選択コマンドのトレースに使う\file{ptrace}パッケージ
と\file{tracefnt}パッケージ、過去の\pLaTeX{}のエミュレートに使う
\file{platex\-release}パッケージと\file{latex\-release}パッケージも、
これと同様の関係になっています。}。


\section{その他の変更点}
\pLaTeX{}の概要については\file{platex.pdf}を、実際のコードは\file{pldoc.pdf}を
参照してください。コードの変更履歴も\file{pldoc.pdf}の末尾で確認できます。


\section{開発版とバグレポート先}
コミュニティ版\pLaTeX{}とu\pLaTeX{}はアスキー版\pLaTeX{}とは異なります
ので、バグレポートはアスキー宛てではなく、日本語\TeX{}開発コミュニティ
に報告してください。\TeX\ ForumやGitHubのIssueシステムが利用できます。
\begin{itemize}
\item \texttt{https://github.com/texjporg/platex}
\item \texttt{https://github.com/texjporg/uplatex}
\end{itemize}

\end{document}
