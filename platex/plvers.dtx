% \iffalse meta-comment
%% File: plvers.dtx
%
%  Copyright 1995-2006 ASCII Corporation.
%  Copyright (c) 2010 ASCII MEDIA WORKS
%  Copyright (c) 2016-2020 Japanese TeX Development Community
%
%  This file is part of the pLaTeX2e system (community edition).
%  -------------------------------------------------------------
%
% \fi
%
% \iffalse
%<*driver>
\ifx\JAPANESEtrue\undefined
  \expandafter\newif\csname ifJAPANESE\endcsname
  \JAPANESEtrue
\fi
\def\eTeX{$\varepsilon$-\TeX}
\def\pTeX{p\kern-.15em\TeX}
\def\epTeX{$\varepsilon$-\pTeX}
\def\pLaTeX{p\kern-.05em\LaTeX}
\def\pLaTeXe{p\kern-.05em\LaTeXe}
%</driver>
% \fi
%
% \setcounter{StandardModuleDepth}{1}
% \StopEventually{}
%
% \iffalse
% \changes{v1.0}{1995/05/16}{p\LaTeXe\ 用に\file{ltvers.dtx}を修正}
% \changes{v1.0a}{1995/08/30}{\LaTeX\ \texttt{!<1995/06/01!>}版用に修正}
% \changes{v1.0b}{1996/01/31}{\LaTeX\ \texttt{!<1995/12/01!>}版用に修正}
% \changes{v1.0c}{1997/01/11}{\LaTeX\ \texttt{!<1996/06/01!>}版用に修正}
% \changes{v1.0d}{1997/01/23}{\LaTeX\ \texttt{!<1996/12/01!>}版用に修正}
% \changes{v1.0e}{1997/07/02}{\LaTeX\ \texttt{!<1997/06/01!>}版用に修正}
% \changes{v1.0f}{1998/02/17}{\LaTeX\ \texttt{!<1997/12/01!>}版用に修正}
% \changes{v1.0g}{1998/09/01}{\LaTeX\ \texttt{!<1998/06/01!>}版用に修正}
% \changes{v1.0h}{1999/04/05}{\LaTeX\ \texttt{!<1998/12/01!>}版用に修正}
% \changes{v1.0i}{1999/08/09}{\LaTeX\ \texttt{!<1999/06/01!>}版用に修正}
% \changes{v1.0j}{2000/02/29}{\LaTeX\ \texttt{!<1999/12/01!>}版用に修正}
% \changes{v1.0k}{2000/11/03}{\LaTeX\ \texttt{!<2000/06/01!>}版用に修正}
% \changes{v1.0l}{2001/09/04}{\LaTeX\ \texttt{!<2001/06/01!>}版用に修正}
% \changes{v1.0m}{2004/08/10}{\LaTeX\ \texttt{!<2003/12/01!>}版対応確認}
% \changes{v1.0n}{2005/01/04}{plfonts.dtxバグ修正}
% \changes{v1.0o}{2006/01/04}{plfonts.dtxバグ修正}
% \changes{v1.0p}{2006/06/27}{plfonts.dtx \LaTeX\ \texttt{!<2005/12/01!>}対応}
% \changes{v1.0q}{2006/11/10}{plfonts.dtxバグ修正}
% \changes{v1.0r}{2016/01/26}{plcore.dtx p\TeX\ (r28720)対応}
% \changes{v1.0s}{2016/02/01}{\LaTeX\ \texttt{!<2015/01/01!>}のlatexreleaseに
%    対応するためのコードを導入}
% \changes{v1.0t}{2016/02/03}{\cs{plIncludeInRelease}と
%    \cs{plEndIncludeInRelease}を新設。}
% \changes{v1.0u}{2016/04/17}{\LaTeX\ \texttt{!<2016/03/31!>}版対応確認}
% \changes{v1.0v}{2016/05/07}{パッチファイルをロードするのをやめた。}
% \changes{v1.0v}{2016/05/07}{起動時の文字列を最新の\LaTeX{}に合わせた。}
% \changes{v1.0w}{2016/05/12}{起動時の文字列に入れる\LaTeX{}のバージョンを
%    元の\LaTeX{}のバナーから引き継ぐように改良}
% \changes{v1.0w}{2016/05/12}{起動時の文字列に入れるBabelのバージョンを
%    元の\LaTeX{}のバナーから取得するコードを\file{platex.ini}から取り入れた}
% \changes{v1.0x}{2016/06/19}{パッチレベルを\file{plvers.dtx}で設定}
% \changes{v1.0y}{2016/06/27}{\file{platex.cfg}の読み込みを追加}
% \changes{v1.0z}{2016/08/26}{\file{platex.cfg}の読み込みを
%    \file{plcore.ltx}から\file{platex.ltx}へ移動}
% \changes{v1.1}{2016/09/14}{起動時のバナーを取得するコードを改良}
% \changes{v1.1a}{2017/02/20}{\LaTeX\ \texttt{!<2017/01/01!>}版対応確認}
% \changes{v1.1b}{2017/03/19}{\cs{l@nohyphenation}の定義を保証
%    (sync with ltfinal.dtx 2017/03/09 v2.0t)}
% \changes{v1.1b}{2017/03/19}{\cs{document@default@language}の定義を保証
%    (sync with ltfinal.dtx 2017/03/09 v2.0t)}
% \changes{v1.1c}{2017/04/23}{\LaTeX\ \texttt{!<2017-04-15!>}版対応確認}
% \changes{v1.1d}{2017/09/24}{パッチレベルが負の数の場合をpre-release扱いへ}
% \changes{v1.1e}{2017/11/09}{\file{latexrelease}と
%    \cs{platexrelease}のエミュレート内部処理を分離}
% \changes{v1.1f}{2017/11/11}{\LaTeX{}のバナーを保存するコードを
%    \file{platex.ltx}から\file{plcore.ltx}へ移動}
% \changes{v1.1g}{2017/12/04}{\pLaTeX{}のバナーの定義時に
%    \cs{pfmtname}, \cs{pfmtversion}, \cs{ppatch@level}を展開しないように}
% \changes{v1.1h}{2018/01/10}{Modify \cs{plIncludeInRelease} code
%    to check matching \cs{plEndIncluderelease}
%    (sync with ltvers.dtx 2018/01/08 v1.1a)}
% \changes{v1.1i}{2018/03/31}{\LaTeXe\ 2017-04-15以降必須}
% \changes{v1.1j}{2018/04/07}{\LaTeX\ \texttt{!<2018-04-01!>}版対応確認}
% \changes{v1.1k}{2018/04/08}{バナー調節のコードを最後(plfinal)ではなく
%    最初(plcore)に早めた}
% \changes{v1.1l}{2018/04/09}{バナーの保存しかたを改良}
% \changes{v1.1l}{2018/04/09}{バナーの再構築のしかたを改良}
% \changes{v1.1m}{2018/09/24}{バナーの再構築を簡略化}
% \changes{v1.1n}{2018/10/31}{\LaTeXe{}とp\LaTeXe{}の更新タイミングずれ対策を
%    \file{plvers.dtx} (plfinal) から\file{plcore.dtx}へ移動}
% \changes{v1.1o}{2018/12/01}{\LaTeX\ \texttt{!<2018-12-01!>}版対応確認}
% \changes{v1.1p}{2019/09/16}{エラーメッセージを更新
%    (sync with ltvers.dtx 2019/07/01 v1.1c)}
% \changes{v1.1q}{2019/10/01}{\LaTeX\ \texttt{!<2019-10-01!>}版対応確認}
% \changes{v1.1r}{2020/02/01}{\LaTeX\ \texttt{!<2020-02-02!>}版対応確認}
% \changes{v1.1s}{2020/03/14}{\LaTeX\ \texttt{!<2020-02-02!> PL5}版対応確認}
% \changes{v1.1t}{2020/03/25}{バナーの再構築を効率化}
% \changes{v1.1u}{2020/03/28}{latexrelease利用時の警告を早めた}
% \changes{v1.1v}{2020/09/28}{新しいフックを活用}
% \changes{v1.1w}{2020/09/30}{\LaTeX\ \texttt{!<2020-10-01!>}版対応確認}
% \changes{v1.1x}{2020/10/07}{フックシステムが利用可能かどうか判定}
% \fi
%
% \iffalse
%<*driver>
% \fi
\ProvidesFile{plvers.dtx}[2020/10/07 v1.1x pLaTeX Kernel (Version Info)]
% \iffalse
\documentclass{jltxdoc}
\GetFileInfo{plvers.dtx}
\author{Ken Nakano \& Hideaki Togashi}
\title{\filename}
\date{作成日:\filedate}
\begin{document}
  \maketitle
  \DocInput{\filename}
\end{document}
%</driver>
% \fi
%
% \section{p\LaTeXe{}のバージョンの設定}
%
% 現在のp\LaTeXe{}がベースとした\LaTeXe{}のバージョンは、下記のとおりです。
% \changes{v1.0}{1995/05/16}{p\LaTeXe\ 用に\file{ltvers.dtx}を修正}
% \changes{v1.0a}{1995/08/30}{\LaTeX\ \texttt{!<1995/06/01!>}版用に修正}
% \changes{v1.0b}{1996/01/31}{\LaTeX\ \texttt{!<1995/12/01!>}版用に修正}
% \changes{v1.0c}{1997/01/11}{\LaTeX\ \texttt{!<1996/06/01!>}版用に修正}
% \changes{v1.0d}{1997/01/23}{\LaTeX\ \texttt{!<1996/12/01!>}版用に修正}
% \changes{v1.0e}{1997/07/02}{\LaTeX\ \texttt{!<1997/06/01!>}版用に修正}
% \changes{v1.0f}{1998/02/17}{\LaTeX\ \texttt{!<1997/12/01!>}版用に修正}
% \changes{v1.0g}{1998/09/01}{\LaTeX\ \texttt{!<1998/06/01!>}版用に修正}
% \changes{v1.0h}{1999/04/05}{\LaTeX\ \texttt{!<1998/12/01!>}版用に修正}
% \changes{v1.0i}{1999/08/09}{\LaTeX\ \texttt{!<1999/06/01!>}版用に修正}
% \changes{v1.0j}{2000/02/29}{\LaTeX\ \texttt{!<1999/12/01!>}版用に修正}
% \changes{v1.0k}{2000/11/03}{\LaTeX\ \texttt{!<2000/06/01!>}版用に修正}
% \changes{v1.0l}{2001/09/04}{\LaTeX\ \texttt{!<2001/06/01!>}版用に修正}
% \changes{v1.0m}{2004/08/10}{\LaTeX\ \texttt{!<2003/12/01!>}版対応確認}
% \changes{v1.0s}{2016/02/01}{\LaTeX\ \texttt{!<2015/01/01!>}版用に修正}
% \changes{v1.0u}{2016/04/17}{\LaTeX\ \texttt{!<2016/03/31!>}版対応確認}
% \changes{v1.1a}{2017/02/20}{\LaTeX\ \texttt{!<2017/01/01!>}版対応確認}
% \changes{v1.1c}{2017/04/23}{\LaTeX\ \texttt{!<2017-04-15!>}版対応確認}
% \changes{v1.1j}{2018/04/07}{\LaTeX\ \texttt{!<2018-04-01!>}版対応確認}
% \changes{v1.1o}{2018/12/01}{\LaTeX\ \texttt{!<2018-12-01!>}版対応確認}
% \changes{v1.1q}{2019/10/01}{\LaTeX\ \texttt{!<2019-10-01!>}版対応確認}
% \changes{v1.1r}{2020/02/01}{\LaTeX\ \texttt{!<2020-02-02!>}版対応確認}
% \changes{v1.1s}{2020/03/14}{\LaTeX\ \texttt{!<2020-02-02!> PL5}版対応確認}
% \changes{v1.1w}{2020/09/30}{\LaTeX\ \texttt{!<2020-10-01!>}版対応確認}
%    \begin{macrocode}
%<*2ekernel>
%\def\fmtname{LaTeX2e}
%\edef\fmtversion
%</2ekernel>
%<latexrelease>\edef\latexreleaseversion
%<platexrelease>\edef\p@known@latexreleaseversion
%<*2ekernel|latexrelease|platexrelease>
   {2020-10-01}
%</2ekernel|latexrelease|platexrelease>
%    \end{macrocode}
%
% また、現在のp\LaTeXe{}は最低でも\LaTeXe\ 2017-04-15以降
% (バージョン番号すなわち日付が|YYYY/MM/DD|形式から
% |YYYY-MM-DD|形式に変更された版)を前提とします。
% なお、\LaTeXe\ 2017/01/01以降はe-\TeX{}必須になっています。
% \changes{v1.1i}{2018/03/31}{\LaTeXe\ 2017-04-15以降必須}
%    \begin{macrocode}
%<*plcore>
\ifx\fmtversion\@undefined
    \errhelp{Please reinstall LaTeX.}%
    \errmessage{This cannot happen!^^JYour file `latex.ltx'
                might be broken}\@@end
\else
  \ifnum\expandafter\@parse@version\fmtversion//00\@nil<20170415
    \errhelp{Please update your TeX installation; if not available,
             obtain it^^Jmanually from CTAN
             (https://ctan.org/pkg/latex-base) or from^^JGitHub
             (https://github.com/latex3/latex2e).}%
    \errmessage{This version of pLaTeX2e requires LaTeX2e 2017-04-15
                or newer!^^JObtain a newer version of `latex',
                otherwise pLaTeX2e setup will^^Jnever succeed}\@@end
  \fi
\fi
%</plcore>
%    \end{macrocode}
%
% \begin{macro}{\pfmtname}
% \begin{macro}{\pfmtversion}
% \begin{macro}{\ppatch@level}
% p\LaTeXe{}のフォーマットファイル名とバージョンを定義します。
% \changes{v1.0x}{2016/06/19}{パッチレベルを\file{plvers.dtx}で設定}
%    \begin{macrocode}
%<*plcore>
\def\pfmtname{pLaTeX2e}
\def\pfmtversion
%</plcore>
%<platexrelease>\edef\platexreleaseversion
%<*plcore|platexrelease>
   {2020-10-01}
%</plcore|platexrelease>
%<*plcore>
\def\ppatch@level{2}
%</plcore>
%    \end{macrocode}
% \end{macro}
% \end{macro}
% \end{macro}
%
% コミュニティ版\pLaTeXe{}ではパッチファイルを使用しないので、
% パッチファイルをロードするコードは削除しました。
% \changes{v1.0v}{2016/05/07}{パッチファイルをロードするのをやめた。}
%
% \iffalse
% 次の部分は、p\LaTeXe{}のパッチファイルをロードするためのコードです。
% バグを修正するためのパッチを配布するかもしれません。
%    \begin{macrocode}
%<*plfinal>
%\IfFileExists{plpatch.ltx}
%  {\typeout{***********************************^^J%
%            * Applying patch file plpatch.ltx *^^J%
%            ***********************************}
%  \def\pfmtversion@topatch{unknown}
%  \input{plpatch.ltx}
%  \ifx\pfmtversion\pfmtversion@topatch
%    \ifx\ppatch@level\@undefined
%      \typeout{^^J^^J^^J%
%   !!!!!!!!!!!!!!!!!!!!!!!!!!!!!!!!!!!!!!!!!!!!!!!!!!!!!!!^^J%
%   !! Patch file `plpatch.ltx' (for version <\pfmtversion@topatch>)^^J%
%   !! is not suitable for version <\pfmtversion> of pLaTeX.^^J^^J%
%   !! Please check if iniptex found an old patch file:^^J%
%   !! --- if so, rename it or delete it, and redo the^^J%
%   !!     iniptex run.^^J%
%   !!!!!!!!!!!!!!!!!!!!!!!!!!!!!!!!!!!!!!!!!!!!!!!!!!!!!!!^^J}%
%      \batchmode \@@end
%    \fi
%  \else
%      \typeout{^^J^^J^^J%
%   !!!!!!!!!!!!!!!!!!!!!!!!!!!!!!!!!!!!!!!!!!!!!!!!!!!!!!!^^J%
%   !! Patch file `plpatch.ltx' (for version <\pfmtversion@topatch>)^^J%
%   !! is not suitable for version <\pfmtversion> of pLaTeX.^^J%
%   !!^^J%
%   !! Please check if iniptex found an old patch file:^^J%
%   !! --- if so, rename it or delete it, and redo the^^J%
%   !!     iniptex run.^^J%
%   !!!!!!!!!!!!!!!!!!!!!!!!!!!!!!!!!!!!!!!!!!!!!!!!!!!!!!!^^J}%
%      \batchmode \@@end
%  \fi
%  \let\pfmtversion@topatch\relax
%  }{}
%</plfinal>
%    \end{macrocode}
% \fi
%
% \section{起動時に実行するコード}
%
% \subsection{\LaTeXe{}起動時の実行コードの取得}
% このファイルの直前で\LaTeXe{}の\file{latex.ltx}が読み込まれているはず
% なので、その起動時の実行コード(|\everyjob|トークンの内容)を保存します。
%
% \LaTeXe\ 2018-04-01 patch level~1までは、|\everyjob|が
%\begin{verbatim}
%   \typeout{LaTeX2e version}\typeout{Babel version}
%\end{verbatim}
% だけでしたが、patch level~2以降では
% いくつかのコードが|\everyjob|で遅延実行されるようになっています。
% それらのコードを抽出するため、最初と最後に区切りトークン
% (それぞれ|\platexNILa|と|\platexNILb|)を付けておきます。
% \changes{v1.1f}{2017/11/11}{\LaTeX{}のバナーを保存するコードを
%    \file{platex.ltx}から\file{plcore.ltx}へ移動}
% \changes{v1.1l}{2018/04/09}{バナーの保存しかたを改良}
%    \begin{macrocode}
%<*plcore>
\edef\platexBANNER{\noexpand\platexNILa
                   \the\everyjob % LaTeX banner and delayed codes
                   \noexpand\platexNILb}
%    \end{macrocode}
%
% \subsection{\pLaTeXe{}起動時に実行するコードの構築}
%
% \begin{macro}{\everyjob}
% \LaTeXe{}起動時の実行コードを元に、\pLaTeXe{}用の調整を加えます。
% \changes{v1.0v}{2016/05/07}{起動時の文字列を最新の\LaTeX{}に合わせた。}
% \changes{v1.0w}{2016/05/12}{起動時の文字列に入れる\LaTeX{}のバージョンを
%    元の\LaTeX{}のバナーから引き継ぐように改良}
% \changes{v1.0w}{2016/05/12}{起動時の文字列に入れるBabelのバージョンを
%    元の\LaTeX{}のバナーから取得するコードを\file{platex.ini}から取り入れた}
% \changes{v1.1}{2016/09/14}{起動時のバナーを取得するコードを改良}
% \changes{v1.1d}{2017/09/24}{パッチレベルが負の数の場合をpre-release扱いへ}
% \changes{v1.1g}{2017/12/04}{\pLaTeX{}のバナーの定義時に
%    \cs{pfmtname}, \cs{pfmtversion}, \cs{ppatch@level}を展開しないように}
% \changes{v1.1k}{2018/04/08}{バナー調節のコードを最後(plfinal)ではなく
%    最初(plcore)に早めた}
% \changes{v1.1l}{2018/04/09}{バナーの再構築のしかたを改良}
% \changes{v1.1m}{2018/09/24}{バナーの再構築を簡略化}
% \changes{v1.1t}{2020/03/25}{バナーの再構築を効率化}
%    \begin{macrocode}
\begingroup
%    \end{macrocode}
%
% \pLaTeXe{}のバージョン表示を作ります。
%    \begin{macrocode}
  \ifnum\ppatch@level=0
    \toks2={\pfmtname\space<\pfmtversion>\space}%
  \else\ifnum\ppatch@level>0
    \toks2={\pfmtname\space<\pfmtversion>+\ppatch@level\space}%
  \else
    \toks2={\pfmtname\space<\pfmtversion>-pre\ppatch@level\space}%
  \fi\fi
%    \end{macrocode}
%
% |\everyjob|の内容をパースして
% \begin{itemize}
% \item \LaTeXe{}のバージョン表示の中身(|\typeout{}|の引数)を |#2|
% \item バージョン表示の前に実行されるコードがあれば |#1|
% \item バージョン表示の後に残っているコードがあれば |#3|
% \end{itemize}
% に入れます。2020年時点では |#1| は空、|#3| は欧文inputencのUTF-8化で
% 遅延されたコードに該当します。
% ^^A Babel v3.20まではこの中に|\typeout{Babel version}|も含まれて
% ^^A いましたが、Babel v3.21からはこのバナー表示は無くなったようです。
% そして、\LaTeXe{}のバージョンと\pLaTeXe{}のバージョンを
% まとめて表示するように整形します。
%    \begin{macrocode}
  \edef\platexNILa#1\typeout#2#3\platexNILb{%
    #1\noexpand\typeout{\the\toks2 (based on #2)}#3}
  \global\everyjob\expandafter\expandafter\expandafter{\platexBANNER}%
%    \end{macrocode}
%
% 不要になったマクロ定義は削除しておきます。
%    \begin{macrocode}
\endgroup
\let\platexBANNER=\@undefined
%</plcore>
%    \end{macrocode}
% \end{macro}
%
% ^^A 起動時に\file{platex.cfg}がある場合、それを読み込むようにする
% ^^A コードは、\file{plcore.ltx}から\file{platex.ltx}へ移動しました。
% \changes{v1.0y}{2016/06/27}{\file{platex.cfg}の読み込みを追加}
% \changes{v1.0z}{2016/08/26}{\file{platex.cfg}の読み込みを
%    \file{plcore.ltx}から\file{platex.ltx}へ移動}
%
% ^^A \LaTeXe{}とp\LaTeXe{}の更新タイミングずれ対策
% ^^A (ハイフネーション関連パラメータの定義)は、
% ^^A \file{plvers.dtx}から\file{plcore.dtx}へ移動しました。
% \changes{v1.1b}{2017/03/19}{\cs{l@nohyphenation}の定義を保証
%    (sync with ltfinal.dtx 2017/03/09 v2.0t)}
% \changes{v1.1b}{2017/03/19}{\cs{document@default@language}の定義を保証
%    (sync with ltfinal.dtx 2017/03/09 v2.0t)}
% \changes{v1.1n}{2018/10/31}{\LaTeXe{}とp\LaTeXe{}の更新タイミングずれ対策を
%    \file{plvers.dtx} (plfinal) から\file{plcore.dtx}へ移動}
%
% \subsection{フックシステムが利用可能かどうか}
% \begin{macro}{\pltx@newhook@avail}
% フォーマット作成時(\file{latex.ltx}の読込後すぐ)と、
% platexreleaseパッケージ内(latexreleaseパッケージ読込後すぐ)で
% それぞれ判定する必要があります。
% ^^A 本当は「|\AddToHook| が定義済みかどうか」を使いたいが
% ^^A \begin{itemize}
% ^^A   \item Format date 2020-02-02 or older: undefined
% ^^A   \item Format date 2020-10-01 or newer: available
% ^^A   \item ... under \texttt{latexrelease} rollback: defined but no-op
% ^^A \end{itemize}
% ^^A という特殊なrollbackに対処するため、|\fmtversion|で判定する。
% ^^A 単なるif-トークンは「読み飛ばし」の考慮が面倒なので使わない。
% \changes{v1.1x}{2020/10/07}{フックシステムが利用可能かどうか判定}
%    \begin{macrocode}
%<*plcore|plhookrelease>
\chardef\pltx@newhook@avail=\z@
\@ifl@t@r\fmtversion{2020/10/01}{\chardef\pltx@newhook@avail=\@ne}{}
%</plcore|plhookrelease>
%    \end{macrocode}
% \end{macro}
%
%
% \section{latexreleaseパッケージへの対応}
%
% 最後に、latexreleaseパッケージへの対応です。
% \begin{macro}{\plIncludeInRelease}
% platexreleaseパッケージでは
% |\plIncludeInRelease|...|\plEndIncludeInRelease|のブロックを使います。
% \changes{v1.0t}{2016/02/03}{\cs{plIncludeInRelease}と
%    \cs{plEndIncludeInRelease}を新設。}
% \changes{v1.1e}{2017/11/09}{\file{latexrelease}と
%    \cs{platexrelease}のエミュレート内部処理を分離}
% \changes{v1.1h}{2018/01/10}{Modify \cs{plIncludeInRelease} code
%    to check matching \cs{plEndIncluderelease}
%    (sync with ltvers.dtx 2018/01/08 v1.1a)}
% \changes{v1.1p}{2019/09/16}{エラーメッセージを更新
%    (sync with ltvers.dtx 2019/07/01 v1.1c)}
%
% \iffalse
% 備忘録:\LaTeX{}では\cs{if@includeinrelease}の定義を
% \file{ltvers.dtx}ではなく\file{ltdefns.dtx}に遅らせてある。
% しかし、これは\cs{newif}がその時点で定義されていないからにすぎず、
% \pLaTeX{}では遅らせる必要がない。
% \fi
%
%    \begin{macrocode}
%<*plcore|platexrelease>
\newif\if@plincludeinrelease
\@plincludeinreleasefalse
\def\plIncludeInRelease#1{%
  \if@plincludeinrelease
    \PackageError{platexrelease}
      {mis-matched \string\plIncludeInRelease}%
      {There is an \string\plEndIncludeRelease\space missing}%
    \@plincludeinreleasefalse
  \fi
  \kernel@ifnextchar[%
  {\@plIncludeInRelease{#1}}
  {\@plIncludeInRelease{#1}[#1]}}
%    \end{macrocode}
%
%    \begin{macrocode}
\def\@plIncludeInRelease#1[#2]{\@plIncludeInRele@se{#2}}
%    \end{macrocode}
%
%    \begin{macrocode}
\def\@plIncludeInRele@se#1#2#3{%
  \toks@{[#1] #3}%
  \expandafter\ifx\csname\string#2+\@currname+plIIR\endcsname\relax
    \ifnum\expandafter\@parse@version#1//00\@nil
          >\expandafter\@parse@version\pfmtversion//00\@nil
      \GenericInfo{}{Skipping: \the\toks@}%
     \expandafter\expandafter\expandafter\@gobble@plIncludeInRelease
    \else
      \GenericInfo{}{Applying: \the\toks@}%
      \@plincludeinreleasetrue
      \expandafter\let\csname\string#2+\@currname+plIIR\endcsname\@empty
    \fi
  \else
    \GenericInfo{}{Already applied: \the\toks@}%
    \expandafter\@gobble@plIncludeInRelease
  \fi
}
%    \end{macrocode}
%
%    \begin{macrocode}
\def\plEndIncludeInRelease{%
  \if@plincludeinrelease
    \@plincludeinreleasefalse
  \else
    \PackageError{platexrelease}
      {mis-matched \string\plEndIncludeInRelease}{}%
  \fi}
%    \end{macrocode}
%
%    \begin{macrocode}
\long\def\@gobble@plIncludeInRelease#1\plEndIncludeInRelease{%
  \@plincludeinreleasefalse
  \@check@plIncludeInRelease#1\plIncludeInRelease
    \@check@plIncludeInRelease\@end@check@plIncludeInRelease}
%    \end{macrocode}
%    \begin{macrocode}
\long\def\@check@plIncludeInRelease#1\plIncludeInRelease
  #2#3\@end@check@plIncludeInRelease{%
  \ifx\@check@plIncludeInRelease#2\else
    \PackageError{platexrelease}
      {skipped \string\plIncludeInRelease\space for tag \string#2}{}%
  \fi}
%</plcore|platexrelease>
%    \end{macrocode}
% \end{macro}
%
% \LaTeXe{}が提供するlatexreleaseパッケージが読み込まれていて、
% かつ\pLaTeXe{}が提供するplatexreleaseパッケージが読み込まれていない
% 場合は、巻き戻し機能によって\pLaTeXe{}のコマンドが\LaTeXe{}のコマンドで
% 上書きされ、動作が壊れてしまいますので、警告を出します。
% \changes{v1.0s}{2016/02/01}{latexrelease利用時に警告を出すようにした}
%
% 当初は|\AtBeginDocument|を使って|\@begindocumenthook|の末尾に
% 警告文を入れていましたが、\LaTeXe~2020-02-02以降に付属の
% latexreleaseパッケージで巻き戻すとフックの実行より早い段階
% (具体的には|\process@table|内の|\kanjiprocess@table|実行中)で
% 「|\series@maybe@drop@one@m|が未定義」というエラーが出てしまうので、
% |\process@table|の先頭に警告文を入れます。
% 万が一|\process@table|も巻き戻し対象とされてしまった場合のため、
% |\@begindocumenthook|の先頭にも入れておきます。
% ^^A この場合は|\process@table|が|\romanprocess@table|相当へと
% ^^A 上書きされているはずであり、|\kanjiprocess@table|は
% ^^A 実行されないのでエラーは出ない。
% \changes{v1.1u}{2020/03/28}{latexrelease利用時の警告を早めた}
%
% \LaTeXe~2020-10-01以降では|\process@table|より早く実行される
% フックが用意されたので、これを利用します。
% \changes{v1.1v}{2020/09/28}{新しいフックを活用}
%    \begin{macrocode}
%<*plfinal>
\ifnum\pltx@newhook@avail=\z@
% for LaTeX2e 2020-02-02 PL5 or older
\expandafter\def\expandafter\process@table\expandafter{%
  \expandafter\p@warn@latexrelease\process@table}
\begingroup
\toks@\expandafter{\expandafter\p@warn@latexrelease\@begindocumenthook}
\xdef\@begindocumenthook{\the\toks@}
\endgroup
\else
% for LaTeX2e 2020-10-01 or later
\AddToHook{begindocument/before}{\p@warn@latexrelease}
\fi
%
\def\p@warn@latexrelease{%
  \ifx\latexreleaseversion\@undefined\else
    \ifx\platexreleaseversion\@undefined
      \@latex@warning@no@line{%
        Package latexrelease is loaded.\MessageBreak
        Some patches in pLaTeX2e core may be overwritten.\MessageBreak
        Consider using platexrelease.\MessageBreak
        See platex.pdf for detail}%
    \fi
  \fi
  \let\p@warn@latexrelease\relax
}
%</plfinal>
%    \end{macrocode}
%
% \Finale
%
\endinput
